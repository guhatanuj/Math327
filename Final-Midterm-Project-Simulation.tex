% Options for packages loaded elsewhere
\PassOptionsToPackage{unicode}{hyperref}
\PassOptionsToPackage{hyphens}{url}
%
\documentclass[
]{article}
\usepackage{amsmath,amssymb}
\usepackage{lmodern}
\usepackage{iftex}
\ifPDFTeX
  \usepackage[T1]{fontenc}
  \usepackage[utf8]{inputenc}
  \usepackage{textcomp} % provide euro and other symbols
\else % if luatex or xetex
  \usepackage{unicode-math}
  \defaultfontfeatures{Scale=MatchLowercase}
  \defaultfontfeatures[\rmfamily]{Ligatures=TeX,Scale=1}
\fi
% Use upquote if available, for straight quotes in verbatim environments
\IfFileExists{upquote.sty}{\usepackage{upquote}}{}
\IfFileExists{microtype.sty}{% use microtype if available
  \usepackage[]{microtype}
  \UseMicrotypeSet[protrusion]{basicmath} % disable protrusion for tt fonts
}{}
\makeatletter
\@ifundefined{KOMAClassName}{% if non-KOMA class
  \IfFileExists{parskip.sty}{%
    \usepackage{parskip}
  }{% else
    \setlength{\parindent}{0pt}
    \setlength{\parskip}{6pt plus 2pt minus 1pt}}
}{% if KOMA class
  \KOMAoptions{parskip=half}}
\makeatother
\usepackage{xcolor}
\usepackage[margin=1in]{geometry}
\usepackage{color}
\usepackage{fancyvrb}
\newcommand{\VerbBar}{|}
\newcommand{\VERB}{\Verb[commandchars=\\\{\}]}
\DefineVerbatimEnvironment{Highlighting}{Verbatim}{commandchars=\\\{\}}
% Add ',fontsize=\small' for more characters per line
\usepackage{framed}
\definecolor{shadecolor}{RGB}{248,248,248}
\newenvironment{Shaded}{\begin{snugshade}}{\end{snugshade}}
\newcommand{\AlertTok}[1]{\textcolor[rgb]{0.94,0.16,0.16}{#1}}
\newcommand{\AnnotationTok}[1]{\textcolor[rgb]{0.56,0.35,0.01}{\textbf{\textit{#1}}}}
\newcommand{\AttributeTok}[1]{\textcolor[rgb]{0.77,0.63,0.00}{#1}}
\newcommand{\BaseNTok}[1]{\textcolor[rgb]{0.00,0.00,0.81}{#1}}
\newcommand{\BuiltInTok}[1]{#1}
\newcommand{\CharTok}[1]{\textcolor[rgb]{0.31,0.60,0.02}{#1}}
\newcommand{\CommentTok}[1]{\textcolor[rgb]{0.56,0.35,0.01}{\textit{#1}}}
\newcommand{\CommentVarTok}[1]{\textcolor[rgb]{0.56,0.35,0.01}{\textbf{\textit{#1}}}}
\newcommand{\ConstantTok}[1]{\textcolor[rgb]{0.00,0.00,0.00}{#1}}
\newcommand{\ControlFlowTok}[1]{\textcolor[rgb]{0.13,0.29,0.53}{\textbf{#1}}}
\newcommand{\DataTypeTok}[1]{\textcolor[rgb]{0.13,0.29,0.53}{#1}}
\newcommand{\DecValTok}[1]{\textcolor[rgb]{0.00,0.00,0.81}{#1}}
\newcommand{\DocumentationTok}[1]{\textcolor[rgb]{0.56,0.35,0.01}{\textbf{\textit{#1}}}}
\newcommand{\ErrorTok}[1]{\textcolor[rgb]{0.64,0.00,0.00}{\textbf{#1}}}
\newcommand{\ExtensionTok}[1]{#1}
\newcommand{\FloatTok}[1]{\textcolor[rgb]{0.00,0.00,0.81}{#1}}
\newcommand{\FunctionTok}[1]{\textcolor[rgb]{0.00,0.00,0.00}{#1}}
\newcommand{\ImportTok}[1]{#1}
\newcommand{\InformationTok}[1]{\textcolor[rgb]{0.56,0.35,0.01}{\textbf{\textit{#1}}}}
\newcommand{\KeywordTok}[1]{\textcolor[rgb]{0.13,0.29,0.53}{\textbf{#1}}}
\newcommand{\NormalTok}[1]{#1}
\newcommand{\OperatorTok}[1]{\textcolor[rgb]{0.81,0.36,0.00}{\textbf{#1}}}
\newcommand{\OtherTok}[1]{\textcolor[rgb]{0.56,0.35,0.01}{#1}}
\newcommand{\PreprocessorTok}[1]{\textcolor[rgb]{0.56,0.35,0.01}{\textit{#1}}}
\newcommand{\RegionMarkerTok}[1]{#1}
\newcommand{\SpecialCharTok}[1]{\textcolor[rgb]{0.00,0.00,0.00}{#1}}
\newcommand{\SpecialStringTok}[1]{\textcolor[rgb]{0.31,0.60,0.02}{#1}}
\newcommand{\StringTok}[1]{\textcolor[rgb]{0.31,0.60,0.02}{#1}}
\newcommand{\VariableTok}[1]{\textcolor[rgb]{0.00,0.00,0.00}{#1}}
\newcommand{\VerbatimStringTok}[1]{\textcolor[rgb]{0.31,0.60,0.02}{#1}}
\newcommand{\WarningTok}[1]{\textcolor[rgb]{0.56,0.35,0.01}{\textbf{\textit{#1}}}}
\usepackage{graphicx}
\makeatletter
\def\maxwidth{\ifdim\Gin@nat@width>\linewidth\linewidth\else\Gin@nat@width\fi}
\def\maxheight{\ifdim\Gin@nat@height>\textheight\textheight\else\Gin@nat@height\fi}
\makeatother
% Scale images if necessary, so that they will not overflow the page
% margins by default, and it is still possible to overwrite the defaults
% using explicit options in \includegraphics[width, height, ...]{}
\setkeys{Gin}{width=\maxwidth,height=\maxheight,keepaspectratio}
% Set default figure placement to htbp
\makeatletter
\def\fps@figure{htbp}
\makeatother
\setlength{\emergencystretch}{3em} % prevent overfull lines
\providecommand{\tightlist}{%
  \setlength{\itemsep}{0pt}\setlength{\parskip}{0pt}}
\setcounter{secnumdepth}{-\maxdimen} % remove section numbering
\ifLuaTeX
  \usepackage{selnolig}  % disable illegal ligatures
\fi
\IfFileExists{bookmark.sty}{\usepackage{bookmark}}{\usepackage{hyperref}}
\IfFileExists{xurl.sty}{\usepackage{xurl}}{} % add URL line breaks if available
\urlstyle{same} % disable monospaced font for URLs
\hypersetup{
  pdftitle={Final Midterm Project Simulation},
  pdfauthor={Dr.~Phil},
  hidelinks,
  pdfcreator={LaTeX via pandoc}}

\title{Final Midterm Project Simulation}
\author{Dr.~Phil}
\date{2023-02-27}

\begin{document}
\maketitle

\hypertarget{introduction}{%
\subsection{Introduction}\label{introduction}}

The Purpose of this report is to simulate data for the in-class question
where the fitted regression model is:

\[ \hat{Final} = 11 + 0.53 \bullet Midterm + 1.2 \bullet Project\]

where the Midterm exam is scored from 0 to 100 and the project is scored
from 0 to 30. The question is which predictor variable (Midterm or
Project) has a stronger association with the response (Final).

We saw in class that when Midterm=100 and project=30 (full points for
both), the predicted Final score is 100. We also noted that when Midterm
= 100, it contributes \(0.53(100)=53\) points to the Final score, and
when Project = 30, it contributes \(1.2(30)=36\) points to the Final
score. So it seems obvious that Midterm has a stronger association with
Final.

We made that conclusion, in part, because we were aware of the different
ranges of possible values for Midterm and Project. If we did not have
that information, we might think that Project has a stronger
relationship, because is has a larger coefficient (i.e., slope): 1.2
vs.~0.53. As we begin doing multiple regression with many predictors, we
need to keep this in mind.

One way to deal with this issue is to standardize all of the predictor
variables, that is, put them all on the same scale. We explore some
different options for this in the analysis below. Note that regardless
of how the predictor variables are standardized, the t-value and p-value
for each predictor remain the same.

\hypertarget{set-up-3d-surface-plotting}{%
\subsubsection{Set up 3D surface
plotting}\label{set-up-3d-surface-plotting}}

This code just sets up 3D scatter plot and surface plotting.

\begin{Shaded}
\begin{Highlighting}[]
\NormalTok{lm\_surface }\OtherTok{\textless{}{-}} \ControlFlowTok{function}\NormalTok{(lmfit, }\AttributeTok{n=}\DecValTok{10}\NormalTok{, }\AttributeTok{alpha=}\FloatTok{0.2}\NormalTok{, ...) \{ }
\NormalTok{  coeffs }\OtherTok{=}\NormalTok{ lmfit}\SpecialCharTok{$}\NormalTok{coefficients}
\NormalTok{  f }\OtherTok{=} \ControlFlowTok{function}\NormalTok{ (x1, x2) coeffs[}\DecValTok{1}\NormalTok{] }\SpecialCharTok{+}\NormalTok{ coeffs[}\DecValTok{2}\NormalTok{]}\SpecialCharTok{*}\NormalTok{x1 }\SpecialCharTok{+}\NormalTok{ coeffs[}\DecValTok{3}\NormalTok{]}\SpecialCharTok{*}\NormalTok{x2}
\NormalTok{  ranges }\OtherTok{\textless{}{-}}\NormalTok{ rgl}\SpecialCharTok{:::}\FunctionTok{.getRanges}\NormalTok{()}
\NormalTok{  x }\OtherTok{\textless{}{-}} \FunctionTok{seq}\NormalTok{(ranges}\SpecialCharTok{$}\NormalTok{xlim[}\DecValTok{1}\NormalTok{], ranges}\SpecialCharTok{$}\NormalTok{xlim[}\DecValTok{2}\NormalTok{], }\AttributeTok{length=}\NormalTok{n)}
\NormalTok{  y }\OtherTok{\textless{}{-}} \FunctionTok{seq}\NormalTok{(ranges}\SpecialCharTok{$}\NormalTok{ylim[}\DecValTok{1}\NormalTok{], ranges}\SpecialCharTok{$}\NormalTok{ylim[}\DecValTok{2}\NormalTok{], }\AttributeTok{length=}\NormalTok{n)}
\NormalTok{  z }\OtherTok{\textless{}{-}} \FunctionTok{outer}\NormalTok{(x,y,f)}
  \FunctionTok{surface3d}\NormalTok{(x, y, z, }\AttributeTok{alpha=}\NormalTok{alpha, ...)}
\NormalTok{\}}

\FunctionTok{library}\NormalTok{(rgl)}
\end{Highlighting}
\end{Shaded}

\hypertarget{data-simulation}{%
\subsection{Data Simulation}\label{data-simulation}}

Next, we simulate some data using the equation above. We take the
population parameter values to be:

\[\beta_0=11\] \[\beta_1=0.53\] \[\beta_2=1.20\]

There are different ways to vary this simulation. The ``levers'' are:

\begin{enumerate}
\def\labelenumi{\arabic{enumi}.}
\tightlist
\item
  The range of values for the Midterm
\item
  The range of values for the Project
\item
  The residual standard error
\item
  The sample size
\end{enumerate}

This first data uses what might be considered a typical range of values
for the Midterm (60-100) and the Project (18-30), that is, the possible
scores are between 60\% and 100\% of the total available.

We'll start with a larger sample size and a lower residual standard
error to demonstrate that the estimated coefficients are close to the
population values.

\begin{Shaded}
\begin{Highlighting}[]
\CommentTok{\# Set the simulation seed value, so that we get the same simulated residuals}
\CommentTok{\# each time}

\FunctionTok{set.seed}\NormalTok{ (}\DecValTok{12345}\NormalTok{)}

\NormalTok{n }\OtherTok{=} \DecValTok{300}
\NormalTok{sig.e }\OtherTok{=} \FloatTok{2.5}
\NormalTok{Midterm }\OtherTok{=} \FunctionTok{runif}\NormalTok{ (n, }\DecValTok{60}\NormalTok{, }\DecValTok{100}\NormalTok{)}
\NormalTok{Project }\OtherTok{=} \FunctionTok{runif}\NormalTok{ (n, }\DecValTok{18}\NormalTok{, }\DecValTok{30}\NormalTok{)}
\NormalTok{Final }\OtherTok{=} \DecValTok{11} \SpecialCharTok{+} \FloatTok{0.53} \SpecialCharTok{*}\NormalTok{ Midterm }\SpecialCharTok{+} \FloatTok{1.2} \SpecialCharTok{*}\NormalTok{ Project }\SpecialCharTok{+} \FunctionTok{rnorm}\NormalTok{ (n, }\DecValTok{0}\NormalTok{, sig.e)}

\NormalTok{sim.df }\OtherTok{=} \FunctionTok{data.frame}\NormalTok{ (Midterm, Project, Final)}
\FunctionTok{plot}\NormalTok{ (sim.df)}
\end{Highlighting}
\end{Shaded}

\includegraphics{Final-Midterm-Project-Simulation_files/figure-latex/unnamed-chunk-2-1.pdf}

Fit the model and predict Final when Midterm=100 and Project=30.

\begin{Shaded}
\begin{Highlighting}[]
\NormalTok{m1 }\OtherTok{=} \FunctionTok{lm}\NormalTok{ (Final }\SpecialCharTok{\textasciitilde{}}\NormalTok{ Midterm }\SpecialCharTok{+}\NormalTok{ Project)}
\FunctionTok{summary}\NormalTok{ (m1)}
\end{Highlighting}
\end{Shaded}

\begin{verbatim}
## 
## Call:
## lm(formula = Final ~ Midterm + Project)
## 
## Residuals:
##     Min      1Q  Median      3Q     Max 
## -6.8069 -1.4294  0.0691  1.5833  8.1662 
## 
## Coefficients:
##             Estimate Std. Error t value Pr(>|t|)    
## (Intercept) 11.46967    1.43076   8.016 2.52e-14 ***
## Midterm      0.53883    0.01201  44.858  < 2e-16 ***
## Project      1.15889    0.04382  26.450  < 2e-16 ***
## ---
## Signif. codes:  0 '***' 0.001 '**' 0.01 '*' 0.05 '.' 0.1 ' ' 1
## 
## Residual standard error: 2.473 on 297 degrees of freedom
## Multiple R-squared:  0.9032, Adjusted R-squared:  0.9025 
## F-statistic:  1385 on 2 and 297 DF,  p-value: < 2.2e-16
\end{verbatim}

\begin{Shaded}
\begin{Highlighting}[]
\FunctionTok{predict}\NormalTok{ (m1, }\FunctionTok{list}\NormalTok{ (}\AttributeTok{Midterm=}\DecValTok{100}\NormalTok{, }\AttributeTok{Project=}\DecValTok{30}\NormalTok{), }\AttributeTok{interval =} \StringTok{\textquotesingle{}confidence\textquotesingle{}}\NormalTok{)}
\end{Highlighting}
\end{Shaded}

\begin{verbatim}
##        fit      lwr      upr
## 1 100.1195 99.40172 100.8374
\end{verbatim}

Here's a 3D scatterplot with the fitted regression surface (must be run
interactively to view):

\begin{Shaded}
\begin{Highlighting}[]
\FunctionTok{plot3d}\NormalTok{ (Midterm, Project, Final, }\AttributeTok{type=}\StringTok{"p"}\NormalTok{, }\AttributeTok{col=}\StringTok{"red"}\NormalTok{, }
        \AttributeTok{xlab=}\StringTok{"Midterm"}\NormalTok{, }\AttributeTok{ylab=}\StringTok{"Project"}\NormalTok{, }\AttributeTok{zlab=}\StringTok{"Final"}\NormalTok{)}

\FunctionTok{lm\_surface}\NormalTok{ (m1)}
\end{Highlighting}
\end{Shaded}

Note in the output above that Midterm has a larger t-value and a smaller
p-value, which means it has a stronger statistical relationship with
Final.

\hypertarget{rescale-project-to-match-midterm}{%
\subsubsection{Rescale Project to match
Midterm}\label{rescale-project-to-match-midterm}}

Using the same data set, let's rescale the Project score to be 0 to 100
(from 0 to 30).

\begin{Shaded}
\begin{Highlighting}[]
\NormalTok{Project100 }\OtherTok{=}\NormalTok{ Project }\SpecialCharTok{*} \DecValTok{10}\SpecialCharTok{/}\DecValTok{3}

\NormalTok{m1a }\OtherTok{=} \FunctionTok{lm}\NormalTok{ (Final }\SpecialCharTok{\textasciitilde{}}\NormalTok{ Midterm }\SpecialCharTok{+}\NormalTok{ Project100)}
\FunctionTok{summary}\NormalTok{ (m1a)}
\end{Highlighting}
\end{Shaded}

\begin{verbatim}
## 
## Call:
## lm(formula = Final ~ Midterm + Project100)
## 
## Residuals:
##     Min      1Q  Median      3Q     Max 
## -6.8069 -1.4294  0.0691  1.5833  8.1662 
## 
## Coefficients:
##             Estimate Std. Error t value Pr(>|t|)    
## (Intercept) 11.46967    1.43076   8.016 2.52e-14 ***
## Midterm      0.53883    0.01201  44.858  < 2e-16 ***
## Project100   0.34767    0.01314  26.450  < 2e-16 ***
## ---
## Signif. codes:  0 '***' 0.001 '**' 0.01 '*' 0.05 '.' 0.1 ' ' 1
## 
## Residual standard error: 2.473 on 297 degrees of freedom
## Multiple R-squared:  0.9032, Adjusted R-squared:  0.9025 
## F-statistic:  1385 on 2 and 297 DF,  p-value: < 2.2e-16
\end{verbatim}

\begin{Shaded}
\begin{Highlighting}[]
\FunctionTok{predict}\NormalTok{ (m1a, }\FunctionTok{list}\NormalTok{ (}\AttributeTok{Midterm=}\DecValTok{100}\NormalTok{, }\AttributeTok{Project100=}\DecValTok{100}\NormalTok{), }\AttributeTok{interval =} \StringTok{\textquotesingle{}confidence\textquotesingle{}}\NormalTok{)}
\end{Highlighting}
\end{Shaded}

\begin{verbatim}
##        fit      lwr      upr
## 1 100.1195 99.40172 100.8374
\end{verbatim}

Note that the y-intercept and the Midterm slope are exactly the same as
before (estimate, standard error, t-value and p-value).

Also note that the coefficient (slope) for Project is exactly the
previous coefficient times 3/10: 0.34767 = 1.15887*3/10. And the t-value
and p-value are exactly the same as before.

Also note that the predicted Final score with Midterm=100 and
Project100=100 (Project=30) is exactly the same as before, including the
confidence interval for the prediction.

\hypertarget{standardize-both-predictor-variables-to-have-mean0-and-sd1}{%
\subsection{Standardize both predictor variables to have mean=0 and
sd=1}\label{standardize-both-predictor-variables-to-have-mean0-and-sd1}}

Next we standardize both predictor variables to have a mean of 0 and a
standard deviation of 1.

\begin{Shaded}
\begin{Highlighting}[]
\NormalTok{sim.df}\SpecialCharTok{$}\NormalTok{Midterm.std }\OtherTok{=} \FunctionTok{scale}\NormalTok{ (sim.df}\SpecialCharTok{$}\NormalTok{Midterm)[,}\DecValTok{1}\NormalTok{]}
\NormalTok{sim.df}\SpecialCharTok{$}\NormalTok{Project.std }\OtherTok{=} \FunctionTok{scale}\NormalTok{ (sim.df}\SpecialCharTok{$}\NormalTok{Project)[,}\DecValTok{1}\NormalTok{]}

\FunctionTok{plot}\NormalTok{ (sim.df)}
\end{Highlighting}
\end{Shaded}

\includegraphics{Final-Midterm-Project-Simulation_files/figure-latex/unnamed-chunk-6-1.pdf}

\begin{Shaded}
\begin{Highlighting}[]
\NormalTok{m2 }\OtherTok{=} \FunctionTok{lm}\NormalTok{ (Final }\SpecialCharTok{\textasciitilde{}}\NormalTok{ Midterm.std }\SpecialCharTok{+}\NormalTok{ Project.std, }\AttributeTok{data=}\NormalTok{sim.df)}
\FunctionTok{summary}\NormalTok{ (m2)}
\end{Highlighting}
\end{Shaded}

\begin{verbatim}
## 
## Call:
## lm(formula = Final ~ Midterm.std + Project.std, data = sim.df)
## 
## Residuals:
##     Min      1Q  Median      3Q     Max 
## -6.8069 -1.4294  0.0691  1.5833  8.1662 
## 
## Coefficients:
##             Estimate Std. Error t value Pr(>|t|)    
## (Intercept)  83.3025     0.1428  583.50   <2e-16 ***
## Midterm.std   6.4166     0.1430   44.86   <2e-16 ***
## Project.std   3.7834     0.1430   26.45   <2e-16 ***
## ---
## Signif. codes:  0 '***' 0.001 '**' 0.01 '*' 0.05 '.' 0.1 ' ' 1
## 
## Residual standard error: 2.473 on 297 degrees of freedom
## Multiple R-squared:  0.9032, Adjusted R-squared:  0.9025 
## F-statistic:  1385 on 2 and 297 DF,  p-value: < 2.2e-16
\end{verbatim}

\begin{Shaded}
\begin{Highlighting}[]
\FunctionTok{predict}\NormalTok{ (m2, }\FunctionTok{list}\NormalTok{ (}\AttributeTok{Midterm.std=}\NormalTok{(}\DecValTok{100} \SpecialCharTok{{-}} \FunctionTok{mean}\NormalTok{ (Midterm))}\SpecialCharTok{/}\FunctionTok{sd}\NormalTok{ (Midterm), }
                   \AttributeTok{Project.std=}\NormalTok{(}\DecValTok{30} \SpecialCharTok{{-}} \FunctionTok{mean}\NormalTok{(Project))}\SpecialCharTok{/}\FunctionTok{sd}\NormalTok{ (Project)), }\AttributeTok{interval =} \StringTok{\textquotesingle{}confidence\textquotesingle{}}\NormalTok{)}
\end{Highlighting}
\end{Shaded}

\begin{verbatim}
##        fit      lwr      upr
## 1 100.1195 99.40172 100.8374
\end{verbatim}

In this case, all of the coefficients are different, but the t-values
and p-values for the slopes are still the same. And Midterm still has
the stronger association with Final (larger t-value).

\hypertarget{d-plot-with-the-standardized-data}{%
\subsubsection{3D Plot with the standardized
data}\label{d-plot-with-the-standardized-data}}

Must be run interactively to view.

\begin{Shaded}
\begin{Highlighting}[]
\FunctionTok{plot3d}\NormalTok{ (sim.df}\SpecialCharTok{$}\NormalTok{Midterm.std, sim.df}\SpecialCharTok{$}\NormalTok{Project.std, Final, }\AttributeTok{type=}\StringTok{"p"}\NormalTok{, }\AttributeTok{col=}\StringTok{"red"}\NormalTok{, }
        \AttributeTok{xlab=}\StringTok{"Midterm std"}\NormalTok{, }\AttributeTok{ylab=}\StringTok{"Project std"}\NormalTok{, }\AttributeTok{zlab=}\StringTok{"Final"}\NormalTok{)}
\FunctionTok{lm\_surface}\NormalTok{(m1)}
\end{Highlighting}
\end{Shaded}

\hypertarget{conclusions-from-the-analyses-above}{%
\subsubsection{Conclusions from the analyses
above:}\label{conclusions-from-the-analyses-above}}

\begin{itemize}
\tightlist
\item
  Beware of interpreting the relative sizes of regression coefficients
  when the predictor variables are on different scales.
\item
  The t-value and p-value for each predictor don't change when you
  rescale that predictor, so t-values and/or p-values can be used to
  judge the relative statistical importance across predictor variables.
\end{itemize}

\hypertarget{the-range-of-each-predictor-matters}{%
\section{The range of each predictor
matters}\label{the-range-of-each-predictor-matters}}

Next, we will demonstrate what happens when the Midterm has a narrower
range of values: 80 to 100 instead of 60 to 100.

\begin{Shaded}
\begin{Highlighting}[]
\NormalTok{Midterm2 }\OtherTok{=} \FunctionTok{runif}\NormalTok{ (n, }\DecValTok{80}\NormalTok{, }\DecValTok{100}\NormalTok{)}
\NormalTok{Project2 }\OtherTok{=} \FunctionTok{runif}\NormalTok{ (n, }\DecValTok{18}\NormalTok{, }\DecValTok{30}\NormalTok{)}
\NormalTok{Final2 }\OtherTok{=} \DecValTok{11} \SpecialCharTok{+} \FloatTok{0.53} \SpecialCharTok{*}\NormalTok{ Midterm2 }\SpecialCharTok{+} \FloatTok{1.2} \SpecialCharTok{*}\NormalTok{ Project2 }\SpecialCharTok{+} \FunctionTok{rnorm}\NormalTok{ (n, }\DecValTok{0}\NormalTok{, sig.e)}

\NormalTok{sim.df2 }\OtherTok{=} \FunctionTok{data.frame}\NormalTok{ (Midterm2, Project2, Final2)}
\FunctionTok{plot}\NormalTok{ (sim.df2)}
\end{Highlighting}
\end{Shaded}

\includegraphics{Final-Midterm-Project-Simulation_files/figure-latex/unnamed-chunk-8-1.pdf}

with Midterm having a narrower range, even with the same population
coefficient values, it now appears that Project has a stronger
association with Final.

3D scatterplot and regression fit:

\begin{Shaded}
\begin{Highlighting}[]
\FunctionTok{plot3d}\NormalTok{ (Midterm2, Project2, Final2, }\AttributeTok{type=}\StringTok{"p"}\NormalTok{, }\AttributeTok{col=}\StringTok{"red"}\NormalTok{, }
        \AttributeTok{xlab=}\StringTok{"Midterm"}\NormalTok{, }\AttributeTok{ylab=}\StringTok{"Project"}\NormalTok{, }\AttributeTok{zlab=}\StringTok{"Final"}\NormalTok{)}

\NormalTok{m3 }\OtherTok{=} \FunctionTok{lm}\NormalTok{ (Final2 }\SpecialCharTok{\textasciitilde{}}\NormalTok{ Midterm2 }\SpecialCharTok{+}\NormalTok{ Project2, }\AttributeTok{data=}\NormalTok{sim.df2)}
\FunctionTok{summary}\NormalTok{ (m3)}
\end{Highlighting}
\end{Shaded}

\begin{verbatim}
## 
## Call:
## lm(formula = Final2 ~ Midterm2 + Project2, data = sim.df2)
## 
## Residuals:
##     Min      1Q  Median      3Q     Max 
## -6.2337 -1.5880 -0.0975  1.8123  5.8513 
## 
## Coefficients:
##             Estimate Std. Error t value Pr(>|t|)    
## (Intercept)  8.26125    2.37795   3.474 0.000589 ***
## Midterm2     0.56805    0.02405  23.617  < 2e-16 ***
## Project2     1.16161    0.04086  28.428  < 2e-16 ***
## ---
## Signif. codes:  0 '***' 0.001 '**' 0.01 '*' 0.05 '.' 0.1 ' ' 1
## 
## Residual standard error: 2.444 on 297 degrees of freedom
## Multiple R-squared:  0.8239, Adjusted R-squared:  0.8227 
## F-statistic: 694.6 on 2 and 297 DF,  p-value: < 2.2e-16
\end{verbatim}

\begin{Shaded}
\begin{Highlighting}[]
\FunctionTok{predict}\NormalTok{ (m3, }\FunctionTok{list}\NormalTok{ (}\AttributeTok{Midterm2=}\DecValTok{100}\NormalTok{, }\AttributeTok{Project2=}\DecValTok{30}\NormalTok{), }\AttributeTok{interval =} \StringTok{\textquotesingle{}confidence\textquotesingle{}}\NormalTok{)}
\end{Highlighting}
\end{Shaded}

\begin{verbatim}
##       fit      lwr      upr
## 1 99.9148 99.21601 100.6136
\end{verbatim}

\begin{Shaded}
\begin{Highlighting}[]
\FunctionTok{lm\_surface}\NormalTok{ (m3)}
\end{Highlighting}
\end{Shaded}

Now it appears that Project has a stronger relationship with Final. The
parameter estimates are sill about the same, so we can still make the
argument that Midterm has the strong effect. And it does, if we're
talking about the actual (practical) effect size. Remember that
statistical significance and practical importance are two different
things. Midterm still has the larger practical importance, but it's
effect is less statistically significant, because we have a narrower
range of values (80-100).

\hypertarget{standardize-the-predictor-variables}{%
\subsubsection{Standardize the predictor
variables}\label{standardize-the-predictor-variables}}

\begin{Shaded}
\begin{Highlighting}[]
\NormalTok{sim.df2}\SpecialCharTok{$}\NormalTok{Midterm.std }\OtherTok{=} \FunctionTok{scale}\NormalTok{ (sim.df2}\SpecialCharTok{$}\NormalTok{Midterm2)[,}\DecValTok{1}\NormalTok{]}
\NormalTok{sim.df2}\SpecialCharTok{$}\NormalTok{Project.std }\OtherTok{=} \FunctionTok{scale}\NormalTok{ (sim.df2}\SpecialCharTok{$}\NormalTok{Project2)[,}\DecValTok{1}\NormalTok{]}

\FunctionTok{plot}\NormalTok{ (sim.df)}
\end{Highlighting}
\end{Shaded}

\includegraphics{Final-Midterm-Project-Simulation_files/figure-latex/unnamed-chunk-10-1.pdf}

\begin{Shaded}
\begin{Highlighting}[]
\NormalTok{m4 }\OtherTok{=} \FunctionTok{lm}\NormalTok{ (Final2 }\SpecialCharTok{\textasciitilde{}}\NormalTok{ Midterm.std }\SpecialCharTok{+}\NormalTok{ Project.std, }\AttributeTok{data=}\NormalTok{sim.df2)}
\FunctionTok{summary}\NormalTok{ (m4)}
\end{Highlighting}
\end{Shaded}

\begin{verbatim}
## 
## Call:
## lm(formula = Final2 ~ Midterm.std + Project.std, data = sim.df2)
## 
## Residuals:
##     Min      1Q  Median      3Q     Max 
## -6.2337 -1.5880 -0.0975  1.8123  5.8513 
## 
## Coefficients:
##             Estimate Std. Error t value Pr(>|t|)    
## (Intercept)  87.8199     0.1411  622.41   <2e-16 ***
## Midterm.std   3.3384     0.1414   23.62   <2e-16 ***
## Project.std   4.0183     0.1414   28.43   <2e-16 ***
## ---
## Signif. codes:  0 '***' 0.001 '**' 0.01 '*' 0.05 '.' 0.1 ' ' 1
## 
## Residual standard error: 2.444 on 297 degrees of freedom
## Multiple R-squared:  0.8239, Adjusted R-squared:  0.8227 
## F-statistic: 694.6 on 2 and 297 DF,  p-value: < 2.2e-16
\end{verbatim}

\begin{Shaded}
\begin{Highlighting}[]
\FunctionTok{predict}\NormalTok{ (m4, }\FunctionTok{list}\NormalTok{ (}\AttributeTok{Midterm.std=}\NormalTok{(}\DecValTok{100} \SpecialCharTok{{-}} \FunctionTok{mean}\NormalTok{ (Midterm2)) }\SpecialCharTok{/} \FunctionTok{sd}\NormalTok{ (Midterm2), }
                   \AttributeTok{Project.std=}\NormalTok{(}\DecValTok{30} \SpecialCharTok{{-}} \FunctionTok{mean}\NormalTok{ (Project2)) }\SpecialCharTok{/} \FunctionTok{sd}\NormalTok{ (Project2)), }\AttributeTok{interval =} \StringTok{\textquotesingle{}confidence\textquotesingle{}}\NormalTok{)}
\end{Highlighting}
\end{Shaded}

\begin{verbatim}
##       fit      lwr      upr
## 1 99.9148 99.21601 100.6136
\end{verbatim}

\hypertarget{standardizing-the-predictor-variables-does-not-affect-any-of-these-results}{%
\subsubsection{Standardizing the predictor variables does not affect any
of these
results:}\label{standardizing-the-predictor-variables-does-not-affect-any-of-these-results}}

\begin{itemize}
\tightlist
\item
  Predicted response and its confidence (or prediction) interval
\item
  t-values and p-values on the coefficients
\item
  R\^{}2, R\_adj\^{}2, residual standard error, F-test and its p-value
\end{itemize}

\end{document}
