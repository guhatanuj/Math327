% Options for packages loaded elsewhere
\PassOptionsToPackage{unicode}{hyperref}
\PassOptionsToPackage{hyphens}{url}
%
\documentclass[
]{article}
\usepackage{amsmath,amssymb}
\usepackage{lmodern}
\usepackage{iftex}
\ifPDFTeX
  \usepackage[T1]{fontenc}
  \usepackage[utf8]{inputenc}
  \usepackage{textcomp} % provide euro and other symbols
\else % if luatex or xetex
  \usepackage{unicode-math}
  \defaultfontfeatures{Scale=MatchLowercase}
  \defaultfontfeatures[\rmfamily]{Ligatures=TeX,Scale=1}
\fi
% Use upquote if available, for straight quotes in verbatim environments
\IfFileExists{upquote.sty}{\usepackage{upquote}}{}
\IfFileExists{microtype.sty}{% use microtype if available
  \usepackage[]{microtype}
  \UseMicrotypeSet[protrusion]{basicmath} % disable protrusion for tt fonts
}{}
\makeatletter
\@ifundefined{KOMAClassName}{% if non-KOMA class
  \IfFileExists{parskip.sty}{%
    \usepackage{parskip}
  }{% else
    \setlength{\parindent}{0pt}
    \setlength{\parskip}{6pt plus 2pt minus 1pt}}
}{% if KOMA class
  \KOMAoptions{parskip=half}}
\makeatother
\usepackage{xcolor}
\usepackage[margin=1in]{geometry}
\usepackage{color}
\usepackage{fancyvrb}
\newcommand{\VerbBar}{|}
\newcommand{\VERB}{\Verb[commandchars=\\\{\}]}
\DefineVerbatimEnvironment{Highlighting}{Verbatim}{commandchars=\\\{\}}
% Add ',fontsize=\small' for more characters per line
\usepackage{framed}
\definecolor{shadecolor}{RGB}{248,248,248}
\newenvironment{Shaded}{\begin{snugshade}}{\end{snugshade}}
\newcommand{\AlertTok}[1]{\textcolor[rgb]{0.94,0.16,0.16}{#1}}
\newcommand{\AnnotationTok}[1]{\textcolor[rgb]{0.56,0.35,0.01}{\textbf{\textit{#1}}}}
\newcommand{\AttributeTok}[1]{\textcolor[rgb]{0.77,0.63,0.00}{#1}}
\newcommand{\BaseNTok}[1]{\textcolor[rgb]{0.00,0.00,0.81}{#1}}
\newcommand{\BuiltInTok}[1]{#1}
\newcommand{\CharTok}[1]{\textcolor[rgb]{0.31,0.60,0.02}{#1}}
\newcommand{\CommentTok}[1]{\textcolor[rgb]{0.56,0.35,0.01}{\textit{#1}}}
\newcommand{\CommentVarTok}[1]{\textcolor[rgb]{0.56,0.35,0.01}{\textbf{\textit{#1}}}}
\newcommand{\ConstantTok}[1]{\textcolor[rgb]{0.00,0.00,0.00}{#1}}
\newcommand{\ControlFlowTok}[1]{\textcolor[rgb]{0.13,0.29,0.53}{\textbf{#1}}}
\newcommand{\DataTypeTok}[1]{\textcolor[rgb]{0.13,0.29,0.53}{#1}}
\newcommand{\DecValTok}[1]{\textcolor[rgb]{0.00,0.00,0.81}{#1}}
\newcommand{\DocumentationTok}[1]{\textcolor[rgb]{0.56,0.35,0.01}{\textbf{\textit{#1}}}}
\newcommand{\ErrorTok}[1]{\textcolor[rgb]{0.64,0.00,0.00}{\textbf{#1}}}
\newcommand{\ExtensionTok}[1]{#1}
\newcommand{\FloatTok}[1]{\textcolor[rgb]{0.00,0.00,0.81}{#1}}
\newcommand{\FunctionTok}[1]{\textcolor[rgb]{0.00,0.00,0.00}{#1}}
\newcommand{\ImportTok}[1]{#1}
\newcommand{\InformationTok}[1]{\textcolor[rgb]{0.56,0.35,0.01}{\textbf{\textit{#1}}}}
\newcommand{\KeywordTok}[1]{\textcolor[rgb]{0.13,0.29,0.53}{\textbf{#1}}}
\newcommand{\NormalTok}[1]{#1}
\newcommand{\OperatorTok}[1]{\textcolor[rgb]{0.81,0.36,0.00}{\textbf{#1}}}
\newcommand{\OtherTok}[1]{\textcolor[rgb]{0.56,0.35,0.01}{#1}}
\newcommand{\PreprocessorTok}[1]{\textcolor[rgb]{0.56,0.35,0.01}{\textit{#1}}}
\newcommand{\RegionMarkerTok}[1]{#1}
\newcommand{\SpecialCharTok}[1]{\textcolor[rgb]{0.00,0.00,0.00}{#1}}
\newcommand{\SpecialStringTok}[1]{\textcolor[rgb]{0.31,0.60,0.02}{#1}}
\newcommand{\StringTok}[1]{\textcolor[rgb]{0.31,0.60,0.02}{#1}}
\newcommand{\VariableTok}[1]{\textcolor[rgb]{0.00,0.00,0.00}{#1}}
\newcommand{\VerbatimStringTok}[1]{\textcolor[rgb]{0.31,0.60,0.02}{#1}}
\newcommand{\WarningTok}[1]{\textcolor[rgb]{0.56,0.35,0.01}{\textbf{\textit{#1}}}}
\usepackage{graphicx}
\makeatletter
\def\maxwidth{\ifdim\Gin@nat@width>\linewidth\linewidth\else\Gin@nat@width\fi}
\def\maxheight{\ifdim\Gin@nat@height>\textheight\textheight\else\Gin@nat@height\fi}
\makeatother
% Scale images if necessary, so that they will not overflow the page
% margins by default, and it is still possible to overwrite the defaults
% using explicit options in \includegraphics[width, height, ...]{}
\setkeys{Gin}{width=\maxwidth,height=\maxheight,keepaspectratio}
% Set default figure placement to htbp
\makeatletter
\def\fps@figure{htbp}
\makeatother
\setlength{\emergencystretch}{3em} % prevent overfull lines
\providecommand{\tightlist}{%
  \setlength{\itemsep}{0pt}\setlength{\parskip}{0pt}}
\setcounter{secnumdepth}{-\maxdimen} % remove section numbering
\ifLuaTeX
  \usepackage{selnolig}  % disable illegal ligatures
\fi
\IfFileExists{bookmark.sty}{\usepackage{bookmark}}{\usepackage{hyperref}}
\IfFileExists{xurl.sty}{\usepackage{xurl}}{} % add URL line breaks if available
\urlstyle{same} % disable monospaced font for URLs
\hypersetup{
  pdftitle={Equity and ACT in Colleges},
  pdfauthor={Tanuj Guha},
  hidelinks,
  pdfcreator={LaTeX via pandoc}}

\title{Equity and ACT in Colleges}
\author{Tanuj Guha}
\date{2023-03-07}

\begin{document}
\maketitle

\emph{Abstract}: In this project we are going to investigate the
relationship between ACT scores, and several other factors that affect
an individual's college admissions. The intention of this project is to
estimate how much of the ACT scores of admitted students can be
explained by other factors such as English Proficiency, Tutoring, High
School GPA, among other things.

Let us dive right into the project, and begin by importing the dataset.

Getting the data (appending legible names to dataset from dictionary)

\begin{Shaded}
\begin{Highlighting}[]
\NormalTok{knitr}\SpecialCharTok{::}\NormalTok{opts\_chunk}\SpecialCharTok{$}\FunctionTok{set}\NormalTok{(}\AttributeTok{echo =} \ConstantTok{TRUE}\NormalTok{)}
\FunctionTok{library}\NormalTok{(tidyverse)}
\FunctionTok{library}\NormalTok{(readxl)}
\FunctionTok{library}\NormalTok{(xlsx)}
\FunctionTok{library}\NormalTok{(datadictionary)}
\NormalTok{data }\OtherTok{=} \FunctionTok{read.csv}\NormalTok{(}\StringTok{"adm2021.csv"}\NormalTok{)}
\NormalTok{dictData }\OtherTok{=} \FunctionTok{read\_xlsx}\NormalTok{(}\StringTok{"adm2021Dict.xlsx"}\NormalTok{, }\AttributeTok{sheet=}\DecValTok{2}\NormalTok{)}
\NormalTok{dict }\OtherTok{=} \FunctionTok{data.frame}\NormalTok{(dictData}\SpecialCharTok{$}\NormalTok{varname, dictData}\SpecialCharTok{$}\NormalTok{varTitle)}
\FunctionTok{names}\NormalTok{(data) }\OtherTok{\textless{}{-}}\NormalTok{ dict}\SpecialCharTok{$}\NormalTok{dictData.varTitle[}\FunctionTok{match}\NormalTok{(}\FunctionTok{names}\NormalTok{(data), dict}\SpecialCharTok{$}\NormalTok{dictData.varname)]}
\CommentTok{\#write.xlsx(data, file = "appendedData.xlsx", sheetName="Colleges")}
\end{Highlighting}
\end{Shaded}

Creating the final data set with only the chosen variables.

\begin{Shaded}
\begin{Highlighting}[]
\NormalTok{colleges }\OtherTok{=} \FunctionTok{select}\NormalTok{(data,}\StringTok{\textquotesingle{}Unique identification number of the institution\textquotesingle{}}\NormalTok{)}

\NormalTok{colleges }\OtherTok{\textless{}{-}} \FunctionTok{cbind}\NormalTok{(colleges,}\FunctionTok{select}\NormalTok{(data,}\StringTok{\textquotesingle{}ACT Composite 75th percentile score\textquotesingle{}}\NormalTok{),}\FunctionTok{select}\NormalTok{(data,}\StringTok{\textquotesingle{}Secondary school GPA\textquotesingle{}}\NormalTok{),}\FunctionTok{select}\NormalTok{(data,}\StringTok{\textquotesingle{}Completion of college{-}preparatory program\textquotesingle{}}\NormalTok{),}\FunctionTok{select}\NormalTok{(data,}\StringTok{\textquotesingle{}TOEFL (Test of English as a Foreign Language\textquotesingle{}}\NormalTok{),}\FunctionTok{select}\NormalTok{(data,}\StringTok{\textquotesingle{}Enrolled total\textquotesingle{}}\NormalTok{),}\FunctionTok{select}\NormalTok{(data,}\StringTok{\textquotesingle{}Percent of first{-}time degree/certificate{-}seeking students submitting ACT scores\textquotesingle{}}\NormalTok{),}\FunctionTok{select}\NormalTok{(data,}\StringTok{\textquotesingle{}SAT Evidence{-}Based Reading and Writing 75th percentile score\textquotesingle{}}\NormalTok{),}\FunctionTok{select}\NormalTok{(data,}\StringTok{\textquotesingle{}SAT Math 75th percentile score\textquotesingle{}}\NormalTok{))}

\NormalTok{colleges }\OtherTok{\textless{}{-}}\NormalTok{ colleges }\SpecialCharTok{\%\textgreater{}\%} 
        \FunctionTok{rename}\NormalTok{(}\StringTok{"ID"} \OtherTok{=} \StringTok{"Unique identification number of the institution"}\NormalTok{,}
               \StringTok{"ACT"} \OtherTok{=} \StringTok{"ACT Composite 75th percentile score"}\NormalTok{,}
               \StringTok{"GPA"} \OtherTok{=} \StringTok{"Secondary school GPA"}\NormalTok{,}
               \StringTok{"CollegePrep"} \OtherTok{=} \StringTok{"Completion of college{-}preparatory program"}\NormalTok{,}
               \StringTok{"TOEFL"} \OtherTok{=} \StringTok{"TOEFL (Test of English as a Foreign Language"}\NormalTok{,}
               \StringTok{"NumberEnrolled"} \OtherTok{=} \StringTok{"Enrolled total"}\NormalTok{,}
               \StringTok{"ACTPercentage"} \OtherTok{=} \StringTok{"Percent of first{-}time degree/certificate{-}seeking students submitting ACT scores"}\NormalTok{,}
               \StringTok{"SATWR"} \OtherTok{=} \StringTok{"SAT Evidence{-}Based Reading and Writing 75th percentile score"}\NormalTok{,}
              \StringTok{"SATM"}\OtherTok{=} \StringTok{"SAT Math 75th percentile score"}\NormalTok{)}

\NormalTok{colleges }\OtherTok{=} \FunctionTok{drop\_na}\NormalTok{(colleges)}
\FunctionTok{write.xlsx}\NormalTok{(colleges, }\AttributeTok{file =} \StringTok{"Colleges.xlsx"}\NormalTok{, }\AttributeTok{sheetName=}\StringTok{"Colleges"}\NormalTok{)}
\end{Highlighting}
\end{Shaded}

This marks the end of house keeping data cleaning.

\begin{Shaded}
\begin{Highlighting}[]
\NormalTok{colleges }\OtherTok{=} \FunctionTok{read\_xlsx}\NormalTok{(}\StringTok{"colleges.xlsx"}\NormalTok{)}
\end{Highlighting}
\end{Shaded}

\begin{verbatim}
## New names:
## * `` -> `...1`
\end{verbatim}

\emph{Introduction}: Conceptually speak, the ACT, or any other
standardized Test for that matter, is touted as an indicator of a
person's mastery of high school matter. This mastery is supposed to be a
prerequisite for college coursework to further build upon. However,
indicators are can often be misleading, and for an indicator to be
robust, it has to be concise while being the \emph{least} reductive.

Specifically, in this project, we are investigating the validity of ACT
scores, and looking into factors that might contribute towards a higher
score for a cohort of students. These cohorts are broken down in terms
freshmen who enrolled in the same college.We expect the ACT scores to
follow a normal distribution. WE also recognize that we haven't
\emph{exhaustively} identified all of the factors that determine the ACT
scores, hence we expect the residuals to have a normal distribution,
when plotted against predicted values.

\emph{Data Collection}:

This dataset has 7 predictor variables:  \_\_ Secondary school GPA
(quant)  \_\_ Completion of college-preparatory program (category) 
\_\_ TOEFL (Test of English as a Foreign Language (categorical)  \_\_
Enrolled total (quant)  \_\_ Percent of first-time
degree/certificate-seeking students submitting ACT scores (quant)  \_\_
SAT Evidence-Based Reading and Writing 75th percentile score) (quant) 
\_\_ SAT Math 75th percentile score (quant)

Each of these predictor variables explain, in some part, the 75th
Percentile ACT scores of incoming students in a given college. For the
sake of privacy, the colleges have been anonymized and are represented
by proxies under column `ID'.

Let us look at the distribution of each of the quantitative variables in
this dataset.

\begin{Shaded}
\begin{Highlighting}[]
\NormalTok{numerical }\OtherTok{=} \FunctionTok{cbind}\NormalTok{(colleges}\SpecialCharTok{$}\NormalTok{ACT, colleges}\SpecialCharTok{$}\NormalTok{NumberEnrolled, colleges}\SpecialCharTok{$}\NormalTok{ACTPercentage, colleges}\SpecialCharTok{$}\NormalTok{SATM, colleges}\SpecialCharTok{$}\NormalTok{SATWR)}
\FunctionTok{barplot}\NormalTok{(}\FunctionTok{scale}\NormalTok{(numerical), }\AttributeTok{beside=}\NormalTok{T)}
\end{Highlighting}
\end{Shaded}

\includegraphics{Project-1_files/figure-latex/unnamed-chunk-3-1.pdf}

Some might accuse our approach as one that isn't assiduous, but I'd beg
to disagree. The above graph visually indicated shows that, the only
variable that needs transformation would be the `NumberEnrolled' column.
To arrive at the conclusion, we first created a dataset of all the
numerical columns of from the colleges dataset. Then, we scaled the
columns (scaling a column does not change its skewness). We chose to
scale the columns because that way \emph{all} of the columns can be
represented in the same bar graph. Then, we plotted all of the columns
side by side in a bar plot. We chose a singular bar plot for brevity's
sake, and in our judgement, it was \emph{enough} to indicate only those
columns, where the skew was painfully obvious.

Now let us investigate `NumberEnrolled' in greater detail.

\begin{Shaded}
\begin{Highlighting}[]
\FunctionTok{hist}\NormalTok{(colleges}\SpecialCharTok{$}\NormalTok{NumberEnrolled)}
\end{Highlighting}
\end{Shaded}

\includegraphics{Project-1_files/figure-latex/unnamed-chunk-4-1.pdf}

Yikes! I was going to do a Shapiro-Wilk test, but I think given the
visual evidence, that would be a moot point. So, let us go right to the
log-transformation.

\begin{Shaded}
\begin{Highlighting}[]
\NormalTok{colleges}\SpecialCharTok{$}\NormalTok{logEnrolled }\OtherTok{\textless{}{-}}\NormalTok{ (}\FunctionTok{log}\NormalTok{(colleges}\SpecialCharTok{$}\NormalTok{NumberEnrolled))}
\FunctionTok{hist}\NormalTok{(colleges}\SpecialCharTok{$}\NormalTok{logEnrolled)}
\end{Highlighting}
\end{Shaded}

\includegraphics{Project-1_files/figure-latex/unnamed-chunk-5-1.pdf}

For our purposes, it (the log trnasformation) works!

\end{document}
